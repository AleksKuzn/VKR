% первая часть
\section{Методы получения трехмерных изображений из двумерных}

В ходе выполнения проекта, главным образом, решается задача преобразования двумерных изображений в трехмерные на мобильных устройствах.

В последние годы заметное место в области преобразования и фильтрации изображений занимает задача преобразования двумерных изображений в трехмерные. На сегодняшний день в мире для этого разработаны различные методики, которые позволяют автоматически создавать так называемые «карты глубины»~\cite{depthMap3} (рисунок ~\ref{fig:s-63}) для двумерных изображений, основываясь на свойствах этого изображение и на некоторых предположениях о характере сцены. 

\begin{figure}[H]
	\centering
	\includegraphics[width=0.7\linewidth]{pics/s-63}
	\caption{карта глубины}
	\label{fig:s-63}
\end{figure} В частности, были проведены исследования следующих методов получения трехмерных изображений из двумерных:

\begin{itemize}
	\item Предположение о том, что изображение имеет линейную перспективу;
	\item Предположение о том, что снимок сделан на открытом пространстве и использование модели рассеяния световых лучей в атмосфере;
	\item Обнаружение теней и восстановление по ним карты глубины;
	\item Обнаружение перекрытий объектов на изображении и используя эту информацию восстановлении карты глубины;
	\item Использование моделей пространственных искажений заданных текстур;
	\item Использование билатеральных симметричных шаблонов;
	\item Использование статистических методов для обучения текстурных шаблонов, на различных расстояниях от объектива и другие методы.
\end{itemize}

Практически все эти методы характеризуются достаточно узким характером сцен, которые могут быть реализованы для преобразования в трехмерное изображение. 

Проведенные предварительные исследования показали, что одним из наиболее перспективных методов преобразования изображений в трехмерные считается метод «дефокусировки», который предполагает, что близкие объекты находятся в фокусе, а более удаленные объекты имеют большее размытие. Используя информацию о размытии той или иной точки на изображении можно предположить о том, насколько далеко она находится от объектива. Используя метод дефокусировки можно построить карту глубины для любой макрофотографии. 

Известные методы получения карты глубины, использующие дефокус удаленных от объектива объектов, основаны на следующей цепочке преобразования изображения~\cite{depthMap1}:

\begin{enumerate} 
	\item Обнаружение краев объектов с использованием фильтра Канни.
	\item Для каждой точки найденного края выполняется оценка расстояния, до нее с использованием гауссовского размытия. Таким образом строится так называемая разреженная карта глубины, которая несет информацию о расстоянии до объектива в некоторых точках изображения. 
	\item Разреженная карта глубины с использованием интерполяции превращается в так называемую «плотную карту глубины», которая уже пригодна для построения трехмерного изображения сцены с любого ракурса. 
\end{enumerate}

Реализация мобильного приложения позволяющего оперативно переводить стандартные фотоизображения в 3D-снимки чрезвычайно актуально и востребовано. С другой стороны, на сегодняшний день имеется определенный научный задел по разработке алгоритмов преобразования 2D -3D. В частности известен ряд различных методик, которые позволяют автоматически создавать «карты глубины» для двумерных изображений, основываясь на свойствах этого изображение, и на некоторых предположениях о характере сцены. Например метод дефокусировки позволяет построить карту глубины фотографии. Вместе с тем информации о программно реализованных в том числе мобильных приложений крайне мало. 

Рассмотренный метод преобразования (см. пункты 1-3) предыдущего раздела страдает двумя существенными недостатками\cite{depthMap2}.

\begin{enumerate}
	\item Разреженная карта глубины получается не всегда гладкой, что приводит к ошибкам последующей интерполяции и, в итоге, к ошибкам в карте глубины. 
	\item Окончательная интерполяция для построения требует больших вычислительных затрат и до последнего времени не имела возможности быть встроенной в мобильные приложения. 
\end{enumerate}

Основываясь на этих двух недостатках существующего метода дефокусировки изображения, требуется провести научное исследование в следующих направлениях:

\begin{enumerate}
	\item Адаптивное сглаживание разреженной карты глубины.  С использованием методов кластеризации краев изображения, в протяженные структуры, глубина точек которых должна подчиняться некоторому закону, а не быть случайной от точки к точке, как в существующем методе. 
	\item Найти способ снижения вычислительных затрат для выполнения двумерной интерполяции при построении плотной карты глубины. 
	\item Нам представляется, что для преобразования двумерного видеоролика в трехмерное представление необходимо разработать методику передачи разреженной карты глубины (с учетом сглаживания) от кадра к кадру. 
\end{enumerate}

Таким образом предлагаемый Проект на сегодняшний день актуален и содержит помимо научной алгоритмической составляющей широкие перспективы коммерциализации.

\subsection{Восстановление глубины из одного расфокусированного изображения}

Рассмотрим сложную задачу восстановления глубины из одного расфокусированного изображения. Входное расфокусированное изображение повторно размыто с использованием гауссова ядра, а значение размытости размытия может быть получено из коэффициента градиента между входными и повторно размытыми изображениями. Распространяя количество размытия в крайних положениях на все изображение, можно восстановить всю карту глубины сцены.

Результат восстановления глубины нашего метода. (рисунок~\ref{fig:input}) Большая интенсивность означает большую глубину на всех картах глубины

\begin{figure}[H]
	\centering
	\includegraphics[width=0.4\linewidth]{pics/input}
	\includegraphics[width=0.4\linewidth]{pics/depth_map}
	\caption{Входное изображение и карта глубины}
	\label{fig:input}
\end{figure}

Сосредоточимся на более сложной проблеме восстановления относительной глубины из одного расфокусированного изображения, захваченного некалиброванной обычной камерой. Метод обратной диффузии~\cite{Proc} моделирует размытие дефокусировки в качестве процесса диффузии тепла и использует неоднородную диффузию тепла для оценки размытости размытия в краевых положениях. В отличие от этого, моделируем размытие дефокусировки как размытие 2D Gaussian. Входное изображение повторно размывается с использованием известного гауссовского размытия, и рассчитывается коэффициент градиента между входными и повторно размытыми изображениями. Величина размытия в краевых местоположениях может быть получена из отношения.

Рассмотрим эффективный метод оценки размытия, основанный на гауссовском градиентном соотношении, и показываем, что он устойчив к шуму, неточному расположению краев и помехам от соседних ребер. Без каких-либо изменений в камерах или при использовании дополнительного освещения наш метод позволяет получить карту глубины сцены, используя только одно расфокусированное изображение, снятое обычной камерой. Как показано (рисунок~\ref{fig:input}), этот метод может извлекать карту глубины сцены с довольно высокой степенью точности.

\subsection{Модель дефокусировки}

Оцениваем размытие размытия в местах краев и предполагаем, что края являются ступенчатыми краями. Идеальный край шага может быть смоделирован как:

\begin{equation}\label{eq:1}
f(x)=Au(x)+B
\end{equation}

где $u(x)$ - ступенчатая функция. A и B - амплитуда и смещение края соответственно. Обратите внимание, что ребро расположено в точке $x=0$.

Предположим, что фокус и дефокусировка подчиняются тонкой модели объектива~\cite{Optics}. Когда объект размещается на расстоянии фокусировки $d_f$, все лучи от точки объекта будут сходиться к одной точке датчика, и изображение будет резким. Лучи из точки другого объекта на расстоянии $d$ достигают нескольких точек датчика и приводят к размытому изображению. Размытый рисунок зависит от формы апертуры и называется кругом путаницы (CoC)~\cite{Optics}. Диаметр CoC характеризует величину расфокусировки и может быть записан как

\begin{equation}\label{eq:2}
c=\frac{|d-d_f|}{d}\frac{f_0^2}{N(d_f-f_0)}
\end{equation}

\begin{figure}[H]
	\centering
	\includegraphics[width=1\linewidth]{pics/focus}
	\caption{Тонкая модель объектива}
	\label{fig:focus}
\end{figure}

где $f_0$ и $N$ - фокусное расстояние и номер остановки камеры соответственно. На рисунке~\ref{fig:focus} показаны фокус и расфокусировка для модели тонких линз и как изменяется диаметр круга замешательства с $d$ и $N$, при фиксированном $f_0$ и $d_f$.
Как мы видим, диаметр CoC $c$ является нелинейной монотонно возрастающей функцией расстояния объекта $d$. Размытие дефокусировки может быть смоделировано как свертка острого изображения с функцией распределения точек (PSF). PSF обычно аппроксимируется гауссовой функцией $g(x,\sigma)$, где стандартное отклонение $\sigma=kc$ пропорционально диаметру CoC $c$. Используем $\sigma$ как меру глубины сцены. Следовательно, размытие ребра $i(x)$ можно представить следующим образом,

\begin{equation}\label{eq:3}
i(x)=f(x)\otimes g(x,\sigma)
\end{equation}

\subsection{Оценка размытости}

На рисунке~\ref{fig:blur} показан обзор метода оценки размытия. Граница шага повторно размыта с использованием гауссова ядра с известным стандартным отклонением. затем рассчитывается соотношение между величиной градиента края ступени и ее размытой версией. Это максимальное значение в краевом положении. Используя максимальное значение, можем вычислить величину размытости дефокусировки в местоположении края.

\begin{figure}[H]
	\centering
	\includegraphics[width=1\linewidth]{pics/blur}
	\caption{Оценка размытости, черная штриховая линия обозначает местоположение края}
	\label{fig:blur}
\end{figure}

Используем двумерное изотропное гауссовское ядро для повторного размытия и величину градиента можно вычислить следующим образом:

\begin{equation}\label{eq:4}
||\bigtriangledown i(x,y)||=\sqrt{\bigtriangledown i_x^2+\bigtriangledown i_y^2}
\end{equation}

где $\bigtriangledown i_x$ и $\bigtriangledown i_y$- градиенты вдоль направлений x и y соответственно. Устанавливаем повторное размытие $\sigma_0=1$ и используем детектор края Canny ~\cite{IEEE} для выполнения обнаружения края.
Шкала размытия оценивается в каждом краевом положении, образуя редкую
карта глубины, обозначаемая $\hat{d}(x)$. Однако из-за шума или мягких теней оценки размытия могут содержать некоторые ошибки. Чтобы подавить эти ошибки, применяется совместный двусторонний фильтр ~\cite{ACM} на разреженной карте глубин $\hat{d}(x)$. Выход совместного двустороннего фильтра в каждом краевом положении х определяется как:

\begin{equation}\label{eq:5}
BF(\hat{d}(x))=\frac{1}{W(x)}\sum_{y\in N(x)}G_{\sigma s}(||x-y||)G_{\sigma r}(||I(x)-I(j)||)\hat{d}(y)
\end{equation}

где $W(x)$ - коэффициент нормировки, а $N(x)$ - окрестность точки x, заданная размером пространственного гауссовского фильтра $G_{\sigma s}$. Фильтрация выполняется только по краям. Совместный двусторонний фильтр корректирует некоторые ошибки в разреженной карте глубины и избегать распространения этих ошибок в интерполяции глубины, описанной в следующем разделе.


\subsection{Интерполяция глубины}

Данный метод оценки размытия дает разреженную карту глубин $d(x)$ с оценками глубины в местах краев. Чтобы получить полную карту глубины $d(x)$, необходимо распространить эти значения из местоположений краев на все изображение. Найдем регуляризованное отображение глубины $d(x)$, которое близко к разреженной карте глубин $d(x)$ в каждом краевом положении. Для этих задач обычно используются методы интерполяции с учетом края~\cite{ACM2}. Здесь применяется матирующий лапласиан~\cite{IEEE2} для выполнения интерполяции.
Мы переписываем разреженную карту глубин d (x) и полное отображение глубины d (x) в их векторной форме как d и d. Тогда проблему интерполяции глубины можно сформулировать как минимизирующую следующую функцию стоимости:


\subsection{Тестирование стабильности и качества алгоритма}

\begin{figure}[H]
	\centering
	\includegraphics[width=0.7\linewidth]{pics/comparison}
	\caption{Восстановление глубины на реальных изображениях. Наш метод может работать как на сценах с непрерывной глубиной (изображение тыквы), так и на слоистой глубине (изображение здания) для получения карты глубины с довольно хорошей точностью.}
	\label{fig:comparison}
\end{figure}\

Как показано на рисунке~\ref{fig:comparison}, я тестирую наш метод на некоторых реальных изображениях. В изображении тыквы глубина сцены непрерывно изменяется от нижней к верхней части изображения. Оценочная карта глубины фиксирует непрерывное изменение глубины. В изображении здания сцена в основном содержит три слоя: стены, дом и небо. Наш метод позволяет создавать карты глубин, точно представляющие эти слои глубины. Как видно из результатов, наш метод позволяет точно восстановить глубину сцены из одного расфокусированного изображения. На рисунке~\ref{fig:flower} мы сравниваем наш метод с методом обратной диффузии~\cite{Proc}.

\begin{figure}[H]
	\centering
	\includegraphics[width=1\linewidth]{pics/flower}
	\caption{Сравнение нашего метода с методом обратной диффузии, на примере цветка}
	\label{fig:flower}
\end{figure}\

Метод обратной диффузии создает грубую слоистую карту глубины. В результате этого цветочный слой плохо отделен фоновыми слоями и содержит некоторые оценки погрешности. Напротив, наш метод способен производить более точную и непрерывную карту глубины, а цветочный слой хорошо отделен фоновой травой и деревьями.

\subsection{Оценка достоверности построения карты глубины}

В ходе выполнения проекта будет создана методика количественной оценки достоверности построения карты глубины, которая необходима для корректного преобразования 2D изображения в 3D вид. Эта методика основана на сравнении показаний глубины, полученных в результате работы алгоритмов преобразования 2D –> 3D и достоверных показаний глубины для данной сцены, которые получены в ручном режиме.

Для реализации методики предполагается произвести ручную разметку тестовых цифровых фотографий в количестве не менее 100 штук. Для этого в ручном режиме устанавливаются контрольные точки, с указанием расстояния до них.

Таким образом формируется разреженная карта глубины, с указанием абсолютных значений. Далее, на изображении находится самая дальняя контрольная точка и все расстояния нормируются на расстояние до нее. Таким образом формируется разреженная карта глубины с относительными расстояниями, и, как следствие, (алгоритмически) формируется «эталонная» плотная карта глубины.

Далее, в результате работы различных алгоритмов преобразования изображения из двумерного в трехмерное автоматически вычисляется как разреженная карта глубины для каждого из анализируемых алгоритмов (создаваемого в рамках проекта и конкурирующих аналогов). Далее предполагается вычислять среднеквадратичную ошибку между эталонными значениями для карты глубины и значениями, вычисленными каждым из алгоритмов.

Предполагается, что алгоритм, реализуемый в ходе выполнения проекта даст снижение такой ошибки между вычисленными и эталонными значениями на величину не менее 25\% (ожидаемое значение – порядка 30\%).