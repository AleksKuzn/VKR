В настоящее  время активно развиваются технологии визуализации графической информации в трехмерном виде. В дополнение к существующим устройствам и технологиям съемки макрообъектов я предлагаю создать приложение для «Android» устройств, с помощью которого пользователи смогут выкладывать в социальных сетях и делиться со знакомыми своими фотографиями в объемном виде.

Основным результатом выполнения проекта будет мобильное приложение, предназначенное для преобразования 2D изображений в 3D вид. Приложение будет распространяться с помощью его размещения в Google Play (для Android-устройств). Соответственно, в качестве основных потребителей создаваемой продукции следует рассматривать владельцев мобильных устройств, которые любят использовать свой телефон или планшет в качестве фотоаппарата. Более того, то подмножество этих пользователей, которые, помимо фотографирования, активно обрабатывают свои фото средствами мобильного устройства и активно делятся этими результатами с друзьями посредством соцсетей.

Поэтому главной целью текущей ВКР является исследование	преобразования двумерных изображений в трехмерные и разработке современного, функционального и удобного пользовательского интерфейса.

В связи с этим, передо мной были поставлены следующие задачи:

\begin{enumerate}
	\item Изучение работы алгоритма определения глубины изображения
	\item Тестирование стабильности и качества алгоритма преобразования 2D фотографий в 3D
	\item Программная реализация алгоритма преобразования 2D фотографии в 3D
	\item Приведение библиотеки в приемлемый вид для использования во «внешнем» приложении.
	\item Анализ существующих продуктов со схожим функционалом
	\item Разработка пользовательского интерфейса мобильного приложения
	\item Разработка концепции и архитектуры мобильного приложения, предназначенного для преобразования 2D в 3D
	\item Проектирование структуры приложения
	\item Изучение основ разработки мобильного приложения в среде Android Studio
	\item Создание мобильного приложения

\end{enumerate}