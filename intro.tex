Основным результатом выполнения проекта будет мобильное приложение, предназначенное для преобразования 2D изображений в 3D вид. Приложение будет распространяться с помощью его размещения в Google Play (для Android-устройств). Соответственно, в качестве основных потребителей создаваемой продукции следует рассматривать владельцев мобильных устройств, которые любят использовать свой телефон или планшет в качестве фотоаппарата. Более того, то подмножество этих пользователей, которые, помимо фотографирования, активно обрабатывают свои фото средствами мобильного устройства и активно делятся этими результатами с друзьями посредством соцсетей.

Поэтому главной целью текущей ВКР является исследованию	преобразования двумерных изображений в трехмерные и разработке современного, функционального и удобного пользовательского интерфейса. 

Актуальной проблемой при написании отчёта по НИР является выбор подходящего шаблона, удовлетворяющего требованиям оформления рисунков, таблиц, текста и других элементов документа.

Цель настоящей работы --- продемонстрировать возможности использования издательской системы \LaTeXe\ для оформления отчётных документов, заодно ссылки расставить \cite{SoetaertRJ2010}.

1. Разработка концепции и архитектуры мобильного приложения, предназначенного для преобразования 2D в 3D
2. Изучение основ разработки мобильного приложения в среде Android Studio
3. Проектирование структуры приложения
4. Разработка пользовательского интерфейса мобильного приложения
5. Изучение работы алгоритма определения глубины изображения
6. Программная реализация алгоритма преобразования 2D фотографии в 3D.
7. Тестирование стабильности и качества алгоритма преобразования 2D фотографий в 3D
8. Приведение библиотеки в приемлемый вид для использования во «внешнем» приложении.
